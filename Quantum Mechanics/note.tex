\documentclass[12pt]{ctexart}
\usepackage{xeCJK}
\usepackage{amsmath}
\usepackage{amssymb} % Provides \mathcal and other math symbols
\usepackage{graphicx}
\usepackage{hyperref}
\usepackage{geometry}
\geometry{a4paper,scale=0.8}
\usepackage{fancyhdr}
\usepackage{color}
\usepackage{float}

\begin{document}
动力学上,中心力导致二维轨道运动。让我们先尝试在二维中量化这个问题。哈密顿量 $H=\frac{{P_{r}}^{2}}{2~m}+\frac{L^{2}}{2~m~r^{2}}-\frac{e}{r}。$

没有方向角$\phi$这一维度,$L$是运动常数,具有量子化的值 $l$, 假设 $$<r|{p_{r}}^{2}=-\frac{\hbar^{2}}{r}\partial_{r}r\partial_{r}<r| = (-\hbar^{2})(\frac{1}{r}\partial_{r}+{\partial_{r}}^{2})<r|$$ 

径向薛定谔方程: $$[(\frac{-\hbar^{2}}{2m})(\frac{1}{r}\partial_{r}+{\partial_{r}}^{2})+\frac{\hbar^{2}}{2~m}\frac{l^{2}}{r^{2}}-\frac{e}{r}-E]R(r)=0$$

当 $r \rightarrow \infty$ 时,$P_{r}^{2}R\rightarrow 0$, $R \rightarrow e^{-\lambda r}$;  定义 $\lambda^{2}\equiv\frac{-2mE}{\hbar^{2}}$ (假设 $E<0$ 为束缚态)

引入物理尺度:$\rho=\lambda r$

用 $\rho$ 表示的径向方程: $(\partial_{\rho}^{2}+\frac{1}{\rho}\partial\rho-\frac{l^{2}}{\rho^{2}}+\frac{2me}{\hbar^{2}\lambda}\frac{1}{\rho}-1) R(\rho)=0$ 定义 $\gamma = \frac{me}{\hbar^{2}\lambda}$

假设解的形式为 $\mathcal{R}(\rho)=e^{-\rho}f(\rho)$, 代入得到 $f(\rho)$ 的方程: $(f^{\prime\prime}(\rho)+(\frac{1}{\rho}-2)f^{\prime}(\rho) + (\frac{2\gamma-1}{\rho} - \frac{l^2}{\rho^2})f(\rho)) = 0$

假设 $f(\rho)$ 的级数解形式为 $f(\rho)=\sum_{m=0}^{\infty} a_{m}\rho^{s+m}$

代入级数解得到递推关系(未完全写出): $\sum_{m}a_{m}[(s+m)(s+m-1)+(s+m)-l^{2}]\rho^{s+m-2}-\sum_{m}a_{m}[2(s+m)-(2\gamma-1)]\rho^{s+m-1}=0.$
化简为:$$a_{m}=-\frac{2\gamma-(2l+1)-2(m-1)}{(s+m)^2-l^2}a_{m-1}$$

分析最低阶的系数: $a_0[s(s-1)+s-l^2]=0 \Rightarrow s^{2}-l^{2}=0$。为了解在 $r=0$ 处行为良好,取 $s=l$ (假设 $l \ge 0$)。

假设级数在 $m=m_{max}$ 处截断 (为了得到束缚态解),即 $a_{m}=0$ for $m>m_{max}$。

最高阶(或递推关系)给出截断条件,这要求 $\rho^{s+m_{max}-1}$ 项的系数满足 $2(s+m_{max}) - (2\gamma - 1) = 0$ (假设 $a_{m_{max}} \ne 0$ 且 $a_{m_{max}+1}=0$ )。 这导致 $\gamma=s+m_{max}+\frac{1}{2}$。

能量 $E = \frac{-\hbar^{2}}{2~m}\lambda^{2}=\frac{-\hbar^{2}}{2~m}(\frac{m~e}{\hbar^{2}}\frac{1}{\gamma})^{2}=\frac{-me^{2}}{2~\hbar^{2}}\frac{1}{\gamma^{2}}$。这与实验拟合。 $\gamma$ 的可能值(对应 $s=l$ 和 $m_{max}=0, 1, 2, ...$):$\gamma = l+m_{max}+1/2$。
\section{作图}
\textbf{假设角向波函数已归一化},所求的级数形式波函数应当有$$\int_{0}^{\infty}R^2(\rho)\rho d\rho=1$$
\begin{table}[H]
    \centering
    \caption{Integration Result of Function $e^{-2x}x^n$}
    \begin{tabular}{|l|l|} 
    \hline
    Function           & Value                            \\ 
    \hline
    $e^{-2x}x^{1}$ & $\frac{1}{4}$   \\
   $ e^{-2x}x^{2}$ & $\frac{1}{4} $  \\
   $ e^{-2x}x^{3} $& $\frac{3}{8} $  \\
    $e^{-2x}x^{4} $ & $\frac{3}{4} $  \\
   $ e^{-2x}x^{5}$ & $\frac{15}{8} $ \\
    \hline
    \end{tabular}
    \end{table}
1.基态 $\left(\boldsymbol{Y}=\mathbf{1 / 2}, \mathbf{I}=\mathbf{0}, m_{-} \max =0\right)$
波函数:
$$
R_{1 / 2}(\rho)=2e^{-\rho}
$$

其中,$\rho=\frac{m e}{\hbar^2} \cdot \frac{2}{1} r$ ,对应能量 $E=-\frac{m e^2}{2 \hbar^2} \cdot 4=-\frac{2 m e^2}{\hbar^2}$ 。

2.第一激发态 $(\boldsymbol{\gamma}=\mathbf{3 / 2})$
$\mathrm{I}=0, \mathrm{~m}_{-} \max =1$
波函数:
$$
R_{3 / 2}^{(0)}(\rho)=\frac{4}{17} e^{-\rho}(1-4 \rho)
$$
若$I=1, m_{-} \max =0$
波函数:
$$
R_{3 / 2}^{(1)}(\rho)=\frac{8}{3} e^{-\rho} \rho
$$

能量均为 $E=-\frac{m e^2}{2 \hbar^2} \cdot \frac{4}{9}=-\frac{2 m e^2}{9 \hbar^2}$ 。

3.第二激发态 $(\boldsymbol{\gamma}=\mathbf{5 / 2})$
 $\mathbf{I}=\mathbf{0}, \mathbf{m}_{-} \max =\mathbf{2}$
波函数:
$$
R_{5 / 2}^{(0)}(\rho)=\frac{8}{707} e^{-\rho}\left(1-6 \rho+9 \rho^2\right)
$$
$I=1, m_{-} \max =1$
波函数:
$$
R_{5 / 2}^{(1)}(\rho)=\frac{24}{5} e^{-\rho} \rho\left(1-\frac{2}{3} \rho\right)
$$
 $\mathbf{I}=\mathbf{2}, \mathbf{m}_{-}$max $=\mathbf{0}$
波函数:
$$
R_{5 / 2}^{(2)}(\rho)=\frac{8}{15} e^{-\rho} \rho^2
$$

能量均为 $E=-\frac{m e^2}{2 \hbar^2} \cdot \frac{4}{25}=-\frac{2 m e^2}{25 \hbar^2}$ 。


\end{document}